\documentclass[polish,12pt]{aghthesis}
% \documentclass[english,12pt]{aghthesis} dla pracy w jêzyku angielskim. Uwaga, w przypadku strony tytu³owej zmiana jêzyka dotyczy tylko kolejno¶ci wersji jêzykowych tytu³u pracy. 
% Szablon przystosowany jest do druku dwustronnego. 

\usepackage[utf8x]{inputenc}
\usepackage{url}

\author{Andrzej Sołtysik}

\titlePL{System wspomagający personalizację zadań domowych}
\titleEN{System supporting personalization of homework}

\fieldofstudy{Informatyka}

\supervisor{dr inż. Ewa Olejarz-Mieszaniec}

\date{\the\year}


\begin{document}

\maketitle

\section{\SectionTitleProjectVision}
\label{sec:cel-wizja}
%\emph{Charakterystyka problemu, motywacja projektu (w tym przegl±d
%  istniej±cych rozwi±zañ prowadz±ca do uzasadnienia celu prac),
%  wizja produktu i analiza zagro¿eñ.}  % niniejsza linijka to tylko komentarz, który nale¿y usun±æ

\section{\SectionTitleScope}
\label{sec:zakres-funkcjonalnosci}

\subsection{Charakterystyka użytkownika systemu}

Głównym użytkownikiem w systemie jest nauczyciel. Ma on dostęp do wszystkich przedstawionych w dokumentacji funkcjonalności. Oprócz nauczyciela, w systemie wyróżniam również użytkownika niezalogowanego. 

\subsection{Wymagania funkcjonalne - historie użytkownika}

Wymagania ułożone są priorytetami. Wymagania z najwyższym priorytetem są fundamentem systemu. Wymagania z priorytetem normalnym są konieczne, aby system spełniał zamierzoną funkcję. Niski priorytet oznacza, że wymagania nim oznaczone usprawniają pracę z systemem, lecz bez nich system byłby nadal używalny. Dodatkową sekcją są ``wymagania dodatkowe'', które nie są konieczne do uzyskania skończonego produktu.

\paragraph{Najwyższy priorytet:}
\begin{itemize}
	\item Jako nauczyciel mogę wprowadzać wyniki uczniów z podziałem na kategorie zadań.
	\item Jako nauczyciel mogę otrzymać od systemu sugestie z jakich kategorii zadań poszczególni uczniowie mają zaległości.
	\item Jako nauczyciel mogę konfigurować klasy.
	\begin{itemize}
		\item Mogę dodać i usuwać klasy.
		\item Mogę ustalić nauczany przedmiot w danej klasie.
		\item Mogę dodawać i usuwać uczniów do klas.
		\item Mogę wybrać sposób komunikacji z uczniami - grupowy adres e-mail, indywidualne adresy e-mail lub brak komunikacji przez e-mail.
	\end{itemize}
	\item Jako niezalogowany użytkownik mogę się zarejestrować i zalogować.
\end{itemize}

\paragraph{Normalny priorytet:}
\begin{itemize}
	\item Jako nauczyciel mogę obejrzeć indywidualny profil ucznia.
	\begin{itemize}
		\item Mogę obejrzeć wyniki ucznia z poszczególnych kategorii.
		\item Mogę sprawdzić, czy uczeń poprawia wyniki z kategorii, z których miał zaległości.
	\end{itemize}
	\item Jako nauczyciel mogę wygenerować zadanie domowe korzystając z wbudowanej bazy zadań.
	\begin{itemize}
		\item Mogę wybrać ilość wygenerowanych zadań.
		\item Mogę zmienić pojedyncze zadania wygenerowane przez system - wybierając je z bazy lub wprowadzając ręcznie. 
		\item Mogę wydrukować wygenerowane zadania.
		\item Mogę wysłać wygenerowane zadania uczniom poprzez e-mail.
	\end{itemize}
	\item Jako nauczyciel mogę przeglądać prywatną bazę zadań i dodawać do niej nowe zadania.
	\item Jako nauczyciel mogę obejrzeć podsumowanie klasy.
	\begin{itemize}
		\item Mogę sprawdzić, które tematy sprawiają uczniom największy problem.
		\item Mogę 
	\end{itemize}
\end{itemize}

\paragraph{Niski priorytet:}
\begin{itemize}
	\item Jako nauczyciel mogę użyć wbudowanej bazy zadań do wygenerowania testu.
	\item Jako nauczyciel mogę manualnie ustalić zaległości ucznia w wybranych tematach.
	\item Jako nauczyciel mogę zapisać szkic wprowdzanych wyników.
\end{itemize}

\paragraph{Wymagania dodatkowe:}
\begin{itemize}
	\item Jako nauczyciel mogę wyniki uczniów automatycznie dodać do e-dziennika <tutaj wstawić system e-dziennik jeśli do jakiegoś dostanę dostęp>.
\end{itemize}


\section{\SectionTitleRealizationAspects}
\label{sec:wybrane-aspekty-realizacji}
%\emph{Przyjête za³o¿enia, struktura i zasada dzia³ania systemu,
%  wykorzystane rozwi±zania technologiczne wraz z uzasadnieniem
%  ich wyboru, istotne mechanizmy i zastosowane algorytmy.} % niniejsza linijka to tylko komentarz, który nale¿y usun±æ

\section{\SectionTitleWorkOrganization}
\label{sec:organizacja-pracy}
%\emph{Struktura zespo³u (role poszczególnych osób), krótki opis i
%  uzasadnienie przyjêtej metodyki i/lub kolejno¶ci prac, planowane i
%  zrealizowane etapy prac ze wskazaniem udzia³u poszczególnych
%  cz³onków zespo³u, wykorzystane praktyki i narzêdzia w zarz±dzaniu
%  projektem.}  % niniejsza linijka to tylko komentarz, który nale¿y usun±æ

\section{\SectionTitleResults}
\label{sec:wyniki-projektu}
%\emph{Wskazanie wyników projektu (co konkretnie uda³o siê uzyskaæ:
%  oprogramowanie, dokumentacja, raporty z testów/wdro¿enia, itd.), prezentacja wyników
%  i ocena ich u¿yteczno¶ci (jak zosta³o to zweryfikowane --- np.\ wnioski
%  klienta/u¿ytkownika, zrealizowane testy wydajno¶ciowe, itd.),
%  istniej±ce ograniczenia i propozycje dalszych prac.}  % niniejsza linijka to tylko komentarz, który nale¿y usun±æ

% o ile to mo¿liwe proszê uzywaæ odwo³añ \cite w konkretnych miejscach a nie \nocite

%\nocite{}

\bibliography{bibliografia}

\end{document}
