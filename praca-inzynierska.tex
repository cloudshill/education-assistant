\documentclass[polish,12pt]{aghthesis}
% \documentclass[english,12pt]{aghthesis} dla pracy w jêzyku angielskim. Uwaga, w przypadku strony tytu³owej zmiana jêzyka dotyczy tylko kolejno¶ci wersji jêzykowych tytu³u pracy. 
% Szablon przystosowany jest do druku dwustronnego. 

\usepackage[utf8x]{inputenc}
\usepackage{url}

\author{Andrzej Sołtysik}

\titlePL{System wspomagający personalizację zadań domowych}
\titleEN{System supporting personalization of homework}

\fieldofstudy{Informatyka}

\supervisor{dr inż. Ewa Olejarz-Mieszaniec}

\date{\the\year}


\begin{document}

\maketitle

\section{\SectionTitleProjectVision}
\label{sec:cel-wizja}
%\emph{Charakterystyka problemu, motywacja projektu (w tym przegl±d
%  istniej±cych rozwi±zañ prowadz±ca do uzasadnienia celu prac),
%  wizja produktu i analiza zagro¿eñ.}  % niniejsza linijka to tylko komentarz, który nale¿y usun±æ

\section{\SectionTitleScope}
\label{sec:zakres-funkcjonalnosci}

\subsection{Charakterystyka użytkownika systemu}

Głównym użytkownikiem w systemie jest nauczyciel. Ma on dostęp do wszystkich przedstawionych w dokumentacji funkcjonalności. Oprócz nauczyciela, w systemie wyróżniam również użytkownika niezalogowanego. 

\subsection{Wymagania funkcjonalne - historie użytkownika}

Wymagania ułożone są priorytetami. Dodatkową sekcją są ``wymagania dodatkowe'', które nie są konieczne do uzyskania skończonego produktu. W poniższych wymaganiach, jeśli nie jest wymieniony jawnie, podmiotem jest nauczyciel.

\paragraph{Najwyższy priorytet:}
\begin{itemize}
	\item{Mogę wprowadzać wyniki uczniów z podziałem na kategorie zadań.}
	\item{Mogę otrzymać od systemu sugestie z jakich kategorii zadań poszczególni uczniowie mają zaległości.}
	\item{Mogę dodawać, usuwać i edytować konfigurację klasy.}
	\item{Mogę dodawać i usuwać uczniów do klas.}
	\item{Jako niezalogowany użytkownik mogę się zarejestrować i zalogować.}
\end{itemize}

\paragraph{Normalny priorytet:}
\begin{itemize}
	\item{Mogę zadać zadanie domowe korzystając z wbudowanej bazy zadań.}
	\item{Mogę dodać zadanie do własnej bazy zadań.}
	\item{Mogę obejrzeć statystyki i wykresy zaległości, przerobionego materiału i innych.} 
\end{itemize}

\paragraph{Niski priorytet:}
\begin{itemize}
	\item{Mogę użyć wbudowanej bazy zadań do wygenerowania testu.}
	\item{Mogę manualnie ustalić zaległości poszczególnych uczniów.}
	\item{Mogę zapisać szkic wprowdzanych wyników.}
\end{itemize}

\paragraph{Wymagania dodatkowe:}
\begin{itemize}
	\item{Mogę wyniki uczniów automatycznie dodać do e-dziennika <tutaj wstawić system e-dziennik jeśli do jakiegoś dostanę dostęp>.}
\end{itemize}


\section{\SectionTitleRealizationAspects}
\label{sec:wybrane-aspekty-realizacji}
%\emph{Przyjête za³o¿enia, struktura i zasada dzia³ania systemu,
%  wykorzystane rozwi±zania technologiczne wraz z uzasadnieniem
%  ich wyboru, istotne mechanizmy i zastosowane algorytmy.} % niniejsza linijka to tylko komentarz, który nale¿y usun±æ

\section{\SectionTitleWorkOrganization}
\label{sec:organizacja-pracy}
%\emph{Struktura zespo³u (role poszczególnych osób), krótki opis i
%  uzasadnienie przyjêtej metodyki i/lub kolejno¶ci prac, planowane i
%  zrealizowane etapy prac ze wskazaniem udzia³u poszczególnych
%  cz³onków zespo³u, wykorzystane praktyki i narzêdzia w zarz±dzaniu
%  projektem.}  % niniejsza linijka to tylko komentarz, który nale¿y usun±æ

\section{\SectionTitleResults}
\label{sec:wyniki-projektu}
%\emph{Wskazanie wyników projektu (co konkretnie uda³o siê uzyskaæ:
%  oprogramowanie, dokumentacja, raporty z testów/wdro¿enia, itd.), prezentacja wyników
%  i ocena ich u¿yteczno¶ci (jak zosta³o to zweryfikowane --- np.\ wnioski
%  klienta/u¿ytkownika, zrealizowane testy wydajno¶ciowe, itd.),
%  istniej±ce ograniczenia i propozycje dalszych prac.}  % niniejsza linijka to tylko komentarz, który nale¿y usun±æ

% o ile to mo¿liwe proszê uzywaæ odwo³añ \cite w konkretnych miejscach a nie \nocite

%\nocite{}

\bibliography{bibliografia}

\end{document}
